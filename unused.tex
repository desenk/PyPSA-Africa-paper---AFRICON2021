\textbf{Least cost electrification}

A review of energy system modelling carried out in~\cite{rocco-fumagalli-ea-2021} finds increasing trends towards OS energy modelling (e.g., OnSSET, OSeMOSYS) in Sub-Saharan Africa, highlighting tendencies towards aggregation of certain nationally specific data (such as investment, policy, and production efficiencies) across Sub-Saharan Africa or regionally (e.g., in East Africa). They also draw attention to issues associated with defining and predicting electricity demand for rural regions which homogenise rural energy users as a single user group.

Energy modelling requires assumptions of energy demand so as to ensure appropriate choice and sizing of electrification option. However, issues have been identified with the processes of defining rural electricity demand, with some models homogenising rural energy users as a single user group irrespective of distance from the grid and population size. However, geographical Information Systems (GIS) data is increasingly being utilised to better. The integration of GIS data into demand prediction and projection in models such as OSeMOSYS~\cite{howells-rogner-ea-2011}, e.g., in~\cite{rocco-fumagalli-ea-2021}, offer new potential to better understand demand needs.

Systematic reviews have sought to understand how different modelling tools, including HOMER Energy, Network Planner, and GEOSIM, calculate least cost electrification options for rural regions of Sub-Saharan Africa. Challenges exist with respect to the collection of data for input parameters such as population and electricity demand, especially for remote regions. There are additional complexities modelling decentralised options such as solar PV, diesel generation and hybrid mini-grids,  which require hourly simulations to ensure their economic and technological appropriateness to context~\cite{cader-blechinger-ea-2016}. A set of deficits relating to the use of modelling tools is identified in~\cite{cader-blechinger-ea-2016}, including: HOMER, which doesn’t consider stand-alone systems, Network Planner, which doesn’t model hydrid mini-grids and whose methodology promotes grid connection based on  proximity (regardless of topography or infrastructure), and GEOSIM, which despite incorporating constraints such as available investment and on-grid energy generation does not model solar hybrid mini-grids~\cite{cader-blechinger-ea-2016}.

Agent-Based Modelling (ABM) is increasingly being applied to energy system modelling~\cite{alfaro-miller-2021,riva-colombo-2020}. It is utilised in~\cite{alfaro-miller-2021} to determine least cost options for rural electrification under scenarios of 100\% renewable energy generation. Recently, machine learning was utilised in~\cite{alova-trotter-ea-2021} to predict future electrification pathways and generation mixes in Africa. Further scope has also been identified with respect to developing energy system models for SSA which incorporate different energy storage technologies (see~\cite{musonye-davidsdottir-ea-2020}).